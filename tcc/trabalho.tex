\documentclass[12pt]{article}

% Pacotes necessários
\usepackage[utf8]{inputenc}
\usepackage[brazil]{babel}
\usepackage[T1]{fontenc}
\usepackage{geometry}
\usepackage{setspace}
\usepackage{indentfirst}
\usepackage{titlesec}
\usepackage{fancyhdr}
\usepackage{graphicx}
\usepackage{amsmath}
\usepackage{amsfonts}
\usepackage{amssymb}
\usepackage{float}
\usepackage{hyperref}
\usepackage{listings}
\usepackage{xcolor}

% Configurações de página
\geometry{a4paper, top=3cm, bottom=2cm, left=3cm, right=2cm}

% Configurações de espaçamento
\onehalfspacing
\setlength{\parindent}{1.25cm}

% Configurações de título
\titleformat{\section}{\normalfont\fontsize{14}{16}\bfseries}{\thesection}{1em}{}
\titleformat{\subsection}{\normalfont\fontsize{12}{14}\bfseries}{\thesubsection}{1em}{}

% Configurações do cabeçalho
\pagestyle{fancy}
\fancyhf{}
\rhead{\footnotesize Nome do Aluno}
\lhead{\footnotesize Título do Trabalho}
\rfoot{\footnotesize\thepage}

% Configurações do listings
\lstset{
    language=Go,
    basicstyle=\ttfamily\small,
    keywordstyle=\color{blue},
    commentstyle=\color{green},
    stringstyle=\color{red},
    numbers=left,
    numberstyle=\tiny,
    frame=single,
    breaklines=true,
    showstringspaces=false
}

\title{\textbf{Sistema de Recuperação de Informação para Documentos Legislativos Brasileiros}}
\author{Nome do Aluno \\ \\ \small Orientador: Prof. Dr. Nome do Orientador}
\date{\small\today}

\begin{document}

% Capa
\begin{titlepage}
\centering
\vspace*{2cm}
{\LARGE\textbf{\MakeUppercase{Centro Universitário Senac}}\par}
\vspace{1cm}
{\large\textbf{Santo Amaro}\par}
\vspace{3cm}
{\Large\textbf{Sistema de Recuperação de Informação para Documentos Legislativos Brasileiros}\par}
\vspace{3cm}
{\large\textbf{Nome do Aluno}\par}
\vspace{2cm}
{\large\textbf{Santo Amaro}\par}
\vspace{0.5cm}
{\large\textbf{2025}\par}
\end{titlepage}

% Resumo
\newpage
\thispagestyle{empty}
\vspace*{2cm}
\begin{center}
{\large\textbf{Resumo}} \\
\end{center}
\vspace{0.5cm}

Este trabalho apresenta o desenvolvimento de um sistema de recuperação de informação especializado na busca e indexação de documentos legislativos brasileiros. O sistema utiliza técnicas avançadas de processamento de linguagem natural e algoritmos de ranqueamento como TF-IDF e BM25 para proporcionar resultados relevantes em consultas a proposições legislativas da Câmara dos Deputados.

\vspace{1cm}
\noindent\textbf{Palavras-chave:} Recuperação de Informação, Processamento de Linguagem Natural, Documentos Legislativos, TF-IDF, BM25.

% Sumário
\newpage
\tableofcontents

% Introdução
\newpage
\section{Introdução}

A crescente quantidade de documentos legislativos produzidos diariamente pela Câmara dos Deputados Brasileira demanda soluções eficazes para indexação e recuperação de informações. Este trabalho apresenta um sistema de recuperação de informação desenvolvido especificamente para esse propósito, com foco em documentos legislativos.

\subsection{Objetivo Geral}

Desenvolver um sistema de recuperação de informação para documentos legislativos brasileiros que utilize técnicas avançadas de indexação e ranqueamento de documentos.

\subsection{Objetivos Específicos}

\begin{itemize}
    \item Implementar mecanismos de indexação de documentos legislativos em formato PDF e TXT
    \item Aplicar algoritmos de ranqueamento TF-IDF e BM25 para ordenação de resultados
    \item Comparar a eficácia dos diferentes algoritmos de ranqueamento
    \item Analisar o impacto de n-gramas com saltos na qualidade dos resultados
\end{itemize}

% Revisão de Literatura
\newpage
\section{Revisão de Literatura}

\subsection{Sistemas de Recuperação de Informação}

Sistemas de recuperação de informação (SRI) são sistemas que buscam informação de um corpus de documentos com base em uma consulta do usuário. Um SRI normalmente indexa documentos e registra a localização de cada termo, permitindo buscas rápidas.

\subsection{TF-IDF}

O Term Frequency-Inverse Document Frequency (TF-IDF) é uma técnica estatística usada para refletir a importância de um termo em um documento em relação a uma coleção de documentos.

\subsection{BM25}

O Okapi BM25 é uma função de ranking usada para calcular a relevância de documentos em sistemas de recuperação de informações baseada em consultas booleanas ponderadas.

% Metodologia
\newpage
\section{Metodologia}

\subsection{Arquitetura do Sistema}

O sistema é composto por três componentes principais:

\subsubsection{Downloader}

Componente responsável por coletar documentos legislativos da API da Câmara dos Deputados.

\subsubsection{Conversor}

Componente que converte documentos PDF para texto e aplica limpeza de dados.

\subsubsection{Indexador}

Componente que indexa os documentos para recuperação eficiente.

\subsection{Linguagem e Ferramentas}

O sistema foi desenvolvido na linguagem Go com as seguintes bibliotecas:
\begin{itemize}
    \item \texttt{github.com/PuerkitoBio/goquery} - Para parsing de HTML
    \item \texttt{github.com/bbalet/stopwords} - Para remoção de stopwords
    \item \texttt{gorm.io/gorm} - Para persistência em banco de dados
    \item \texttt{github.com/pdfcpu/pdfcpu} - Para manipulação de PDFs
\end{itemize}

% Resultados
\newpage
\section{Resultados}

\subsection{Implementação}

Foram implementados os três componentes principais: download, conversão e indexação de documentos legislativos.

\subsection{Algoritmos de Ranqueamento}

Implementados os algoritmos TF-IDF e BM25 para ranqueamento de documentos, permitindo comparação de desempenho.

\subsection{Benchmarking}

Realizados testes comparativos entre diferentes configurações de n-gramas e algoritmos de ranqueamento.

% Discussão
\newpage
\section{Discussão}

\subsection{Análise dos Resultados}

Os resultados mostraram que o uso de n-gramas com saltos pode melhorar a precisão da recuperação de informação em documentos legislativos.

\subsection{Limitações}

O sistema atual apresenta algumas limitações, como a necessidade de maior otimização de performance e a falta de interface gráfica para interação do usuário.

% Conclusão
\newpage
\section{Conclusão}

Este trabalho apresentou um sistema de recuperação de informação para documentos legislativos brasileiros. Os resultados iniciais são promissores, mas há espaço para melhorias.

% Referências
\newpage
\begin{thebibliography}{9}

\bibitem{manning2008}
Manning, C. D., Raghavan, P., \& Schütze, H. (2008). \textit{Introduction to Information Retrieval}. Cambridge University Press.

\bibitem{robertson2009}
Robertson, S., \& Zaragoza, H. (2009). \textit{The Probabilistic Relevance Framework: BM25 and Beyond}. Foundations and Trends in Information Retrieval.

\bibitem{go2025}
Golang Documentation. (2025). \textit{Effective Go}. Recuperado de \url{https://golang.org/doc/effective_go}

\end{thebibliography}

\end{document}